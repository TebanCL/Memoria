\begin{glosario}

\begin{itemize}
\item API: acrónimo inglés para Application Programming Interface o Interfaz de programación de aplicaciones.
\item BSON: es un formato de intercambio de datos usado principalmente para su almacenamiento y transferencia en la base de datos MongoDB. Es una representación binaria de estructuras de datos y mapas. El nombre BSON está basado en el término JSON y significa \textit{Binary JSON} (JSON Binario).
\item Computación distribuida: computación haciendo uso de distintos nodos para el procesamiento en lugar de la programación clásica en un único nodo.
\item \textit{Framework}: estructura que sirve de base para la organización y desarrollo del software. Puede proveer de soporte a programas y bibliotecas entro otros.
\item \textit{Hashtag}: secuencia de palabras concatenadas antecedidas por un gato (\#), actúa como etiqueta.
\item \textit{Hot-spot} o cuello de botella: se refiere a nodos donde se producen cuellos de botella producto de que recibe información de muchos otros nodos y donde sólo este puede procesar la información.
\item JSON: acrónimo de \textit{JavaScript Object Notation}, es un formato de texto ligero para el intercambio de datos.
\item JVM: una máquina virtual Java (JVM), es una máquina virtual de proceso nativo, es decir, ejecutable en una plataforma específica, capaz de interpretar y ejecutar instrucciones expresadas en un código binario especial el cual es generado por el compilador del lenguaje Java.
\item \textit{REST}: arquitectura de software presentada por Roy Fielding, para más información refiérase a su tesis doctoral. \cite{Fielding}.
\item \textit{Spike}: actividades de investigación, exploración y prototipado en metodologías ágiles.
\item \textit{Stream}: se refiere al flujo de información, en este contexto, al flujo de mensajes que se están produciendo en cada momento.
\item \textit{Trending Topic}: palabras o frases más repetidas en un momento en concreto.
\item \textit{Tradeoff}: anglicismo. se refiere a que debe sacrificarse algo para mejorar en otro.
\item \textit{Tweet}: los mensajes de \textit{Twitter} son denominados como \textit{tweets}. Tienen una longitud máxima de 140 caracteres. 
\item \textit{Twitter}: servicio de microblogging. Permite enviar mensajes de texto de corta longitud.
\item URL: secuencia de caracteres que siguen un formato estándar que asigna recursos en una red.
\item \textit{Performance}: anglicismo. Se refiere al rendimiento.
\item \textit{Stopwords}: anglicismo. Se traduce como Palabra vacía. Consiste en palabras sin significados como artículos, pronombres, preposiciones, etc. Suelen filtrarse para realizar procesamiento de lenguaje natural.
\end{itemize}

\end{glosario}
