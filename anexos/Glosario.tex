\begin{glosario}

\begin{itemize}
\item API: Acrónimo inglés para Application Programming Interface o Interfaz de programación de aplicaciones.
\item Computación distribuida: Computación haciendo uso de distintos nodos para el procesamiento en lugar de la programación clásica en un único nodo.
\item \textit{Framework}: Estructura que sirve de base para la organización y desarrollo del software. Puede proveer de soporte a programas y bibliotecas entro otros.
\item \textit{Hashtag}: Secuencia de palabras concatenadas antecedidas por un gato (\#), actúa como etiqueta.
\item Cuello de botella o \textit{hot-spot}: Se refiere a nodos donde se producen cuellos de botella producto de que recibe información de muchos otros nodos y donde sólo este puede procesar la información.
\item \textit{REST}: Arquitectura de software presentada por Roy Fielding, para más información refiérase a su tesis doctoral. \cite{Fielding}.
\item \textit{Stream}: Se refiere al flujo de información, en este contexto, al flujo de mensajes que se están produciendo en cada momento.
\item \textit{Trending Topic}: Palabras o frases más repetidas en un momento en concreto.
\item \textit{Tradeoff}: Anglicismo. se refiere a que debe sacrificarse algo para mejorar en otro.
\item \textit{Tweet}: Los mensajes de \textit{Twitter} son denominados como \textit{tweets}. Tienen una longitud máxima de 140 caracteres. 
\item \textit{Twitter}: Servicio de microblogging. Permite enviar mensajes de texto de corta longitud.
\item URL: Secuencia de caracteres que siguen un formato estándar que asigna recursos en una red.
\item \textit{Performance}: Anglicismo. Se refiere al rendimiento.
\item \textit{Stopwords}: Anglicismo. Se traduce como Palabra vacía. Consiste en palabras sin significados como artículos, pronombres, preposiciones, etc. Suelen filtrarse para realizar procesamiento de lenguaje natural.
\end{itemize}

\end{glosario}
