\resumenCastellano{
\textit{Twitter} es una red social que cuenta con millones de usuarios en todo el mundo y, en Chile, alcanza cerca de los 1.700.000 accesos diariamente. Sus usos van desde ser un microblog personal hasta la entrega de información o comunicación entre pares. Es por ello que, en épocas de necesidad, como lo es el periodo inmediato luego de la ocurrencia de una catástrofe natural, las personas tienden a publicar sus experiencias dentro de éste servicio.\\
Teniendo en cuentra lo anterior es que se propone un sistema capaz de recoger la información desde \textit{Twitter} — los denominados \textit{tweets} — y procesarla a fin de detectar si es que un \textit{tweet} corresponde a uno en el que el usuario haga mención alguno de los tipos de necesidad que el sistema será capáz de detectar y, finalmente, hacer uso de la información implícita (contenido del \textit{tweet} o metadatos), para presentar la necesidad como un punto en un mapa geográfico del país, de modo que la información obtenida pueda ser tomada por las autoridades correspondientes para que, de esta forma, facilite el proceso de toma de desiciones en cuanto al envío de ayuda a una determinada área dadas las necesidades expresadas por la población.\\
Para lograr lo expuesto anteriormente, se ha construido un sistema cuya principal característica está en identificar, a partir de la información contenida en un \textit{tweet}, si éste hace referencia o no a una necesidad y a dónde corresponde. 
(\textbf{PRINCIPALES RESULTADOS Y CONCLUSIONES ACA}).

\vspace*{0.5cm}
\KeywordsES{Palabras; Claves}
}

\newpage

\resumenIngles{
\textit{Twitter} is a social network that already has millions of users worldwide and, in Chile, reach about of 1.7 millions of accesses daily. Its uses range from being a personal microblog up to information delivery and comunication between peers. It's because of this that in emergencies, such as the inmediate period after a natural catastrophe, people tends to post their experiences on this service.\\
With this in mind is that is proposed a system able to get information from \textit{Twitter} — as \textit{tweets} — and process it to detect if a user's \textit{tweet} mentions one of the needs that the system can handle and, finally, use the implicit information in the tweet (metadata) and render the need as a geographical position in country's map, thus authority can use the given information and ease the desition making process of sending help to affected areas with the information given by the population.\\
To achieve these statements previously exposed has been made a system whose main characteristics are identify, by the \textit{tweet}'s given metadata, if it references a need and where it corresponds.

\vspace*{0.5cm}
\KeywordsEN{Key; words}
}
