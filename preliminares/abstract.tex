\resumenCastellano{

\textit{Twitter} es una red social que cuenta con millones de usuarios en todo el mundo y, en Chile, alcanza cerca de los 1.700.000 accesos diariamente. Sus usos van desde ser un \textit{microblog} personal hasta la entrega de información o comunicación entre pares. Es por ello que, en épocas de necesidad, como lo es el periodo inmediato luego de la ocurrencia de una catástrofe natural, las personas tienden a publicar sus experiencias dentro de éste servicio.


Teniendo en cuentra lo anterior es que se construyó un sistema, basado en el paradigma de procesamiento de \textit{streams}, capaz de recoger información de manera automática desde \textit{Twitter} — los denominados \textit{tweets} — y procesarlos a fin de detectar si es que un \textit{tweet} corresponde a uno en el que el usuario haga mención alguno de los tipos de necesidad que el sistema es capaz de detectar y, finalmente, hacer uso de la información implícita (contenido del \textit{tweet} o metadatos), para presentar la necesidad como un punto en un mapa geográfico del país, de modo que la información obtenida pueda ser tomada por las autoridades correspondientes para que, de esta forma, facilite el proceso de toma de desiciones en cuanto al envío de ayuda a una determinada área dadas las necesidades expresadas por la población. 

Para lograrlo se hizo uso de programación extrema en conjunto con la metodología KDD para construir una aplicación que haga uso de un clasificador de textos para la detección de necesidades, mediante los cuales se logra ubicar y clasificar correctamente datos obtenidos del terremoto de Concepción ocurrido el 27 de febrero del año 2010. Además se prueba la efectividad del operador detector de ubicación para suplir la carencia de datos de geolocalización en los metadatos de \textit{Twitter}.

Este trabajo se enmarca en el proyecto FONDEF IDeA (Dos etapas) código ID15I10560 en el que participa un equipo de investigación de la universidad.


\vspace*{0.5cm}
\KeywordsES{Programación extrema; KDD; Clasficador; \textit{Twitter}; Detección de necesidades; Geolocalización; \textit{Stream processing}; Clasificador de texto; Redes sociales; Herramienta de apoyo a desastres}
}

\newpage

\resumenIngles{
\textit{Twitter} is a social network that already has millions of users worldwide and, in Chile, reach about of 1.7 millions of accesses daily. Its uses range from being a personal microblog up to information delivery and comunication between peers. It's because of this that in emergencies, such as the inmediate period after a natural catastrophe, people tends to post their experiences on this service.

With this in mind is that is built a system, based on stream processing paradigm, that it's able to get, automatically, information from \textit{Twitter} — as \textit{tweets} — and process it to detect if a user's \textit{tweet} mentions one of the needs that the system can handle and, finally, use the implicit information in the tweet (metadata) and render the need as a geographical position in country's map, thus authority can use the given information and ease the desition making process of sending help to affected areas with the information given by the population.

To achieve these statements previously exposed extreme programming has been used in conjunction with KDD in order to build a text's classifier to detect people's needs. With these two methods was possible to classify and place correctly the data obtained from Concepción's earthquake at February the 27, 2010. It was proved the effectiveness of the proposed location recognizer created to supply the lack of geolocalization data in Twitter's metadata.

This project is within the FONDEF IDeA (Two stages), code: ID15I10560, where a University's research team is working on.

\vspace*{0.5cm}
\KeywordsEN{Extreme programming; KDD; Classifier; Twitter; Detect people's Need; Geolocalization; Stream processing; Text's classification; Social networks; Post-disaster support tool}
}
