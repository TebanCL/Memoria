\resumenCastellano{

\textit{Twitter} es una red social que cuenta con millones de usuarios en todo el mundo y, en Chile, alcanza cerca de los 1.700.000 de accesos diarios. Sus usos van desde ser un \textit{microblog} personal hasta la entrega de información o comunicación entre pares. Es por ello que, en épocas de necesidad, como lo es el periodo inmediato luego de la ocurrencia de una catástrofe natural, las personas tienden a publicar sus experiencias dentro de este servicio.

Teniendo en cuenta lo anterior se construyó un sistema, basado en el paradigma de procesamiento de \textit{streams}, capaz de recoger información de manera automática y en tiempo real desde \textit{Twitter} — los denominados \textit{tweets} — y procesarlos a fin de detectar si es que un \textit{tweet} corresponde a uno en el que el usuario haga mención a alguna necesidad. Para el sistema, una necesidad puede ser de siete tipos: agua (falta de agua en el sector), alimento (falta de alimento en el sector), electricidad (falta de luz eléctrica en el sector), personas (búsqueda o hallazgo de personas), comunicación (entregue o solicite información de una localidad), seguridad (existencia de riesgo para las personas) o irrelevantes (no tienen que ver con ninguna anterior). Para el proceso de detección de necesidades el sistema hace uso de la información implícita (contenido del \textit{tweet} o metadatos), para presentar la necesidad como un punto en un mapa geográfico del país, de modo que la información obtenida pueda ser tomada por las autoridades correspondientes para que facilite el proceso de toma de desiciones en cuanto al envío de ayuda a una determinada área dadas las necesidades expresadas por la población. La construcción de este sistema hizo uso de dos metodologías. Por un lado, programación extrema para el desarrollo del sistema en sí y por otro, KDD para la generación de un modelo de clasificación.

El clasificador está construido utilizando un \textit{dataset} del terremoto de febrero del 2010 del cual se extrajo la información relevante y se etiquetó para ser entregada a la herramienta Mallet quien lo construyó haciendo uso de \textit{Naïve Bayes} logrando un 87\% de precisión en la predicción.

Este trabajo se enmarca en el proyecto FONDEF IDeA (Dos etapas), código ID15I10560 en el que participa un equipo de investigación de la universidad.

\vspace*{0.5cm}
\KeywordsES{Programación extrema; KDD; Clasficador; \textit{Twitter}; Detección de necesidades; Geolocalización; \textit{Stream processing}; Clasificador de texto; Redes sociales; Herramienta de apoyo a desastres}
}

\newpage

\resumenIngles{
\textit{Twitter} is a social network that has millions of users worldwide, only in Chile more than 1.7M of accesses are registered every day. It's use has been extended further than being just a microblog, nowadays, it has been used also as a communication media to spread information on real-time such as news or people search after a disaster.

On this context, this work proposes to build a real-time analysis platform able to detect people’s needs expressed on this social network after the occurrence of a natural disaster such an earthquake. The system is able to recognize and classify tweets under 7 categories: water, food, electricity, missing people, communication, security, and irrelevants. In our particular case, we focus the analysis to tweets emitted on post disaster scenarios. By exploiting the tweet metadata and a classifier, our system is able to detect needs on real time and display it over a geographic map. Our goal is to provide a tool able to ease the decision making process for authorities and improve the help distribution by exploiting social networks information.

Our platform is composed by two modules: front-end and the back-end. The first support the visualization and user interactions, while the latter provide the substrate to process massive flows of data in a scalable manner. The platform was built following two well-known Software development methodologies such as XP and KDD. We test our platform over with a real data set of twees  collected during the earthquake of Chile in February 2010.

This project was developed in the context of the FONDEF IDeA, code: ID15I10560 USACH.

\vspace*{0.5cm}
\KeywordsEN{Extreme programming; KDD; Classifier; Twitter; Detect people's Need; Geolocalization; Stream processing; Text's classification; Social networks; Post-disaster support tool}
}
