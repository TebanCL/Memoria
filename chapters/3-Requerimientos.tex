\chapter{Requerimientos}
\label{cap:Requerimientos}

Éste capítulo detalla los requisitos de las aplicaciones, descritos como historias de usuario. Éstos señalan las necesidades de los clientes.

El cliente es uno de los clientes del proyecto FONDEF IDeA, código ID15I10560, proyecto orientado a generar herramientas para la gestión de desastres naturales. Los requerimientos se levantan en reuniones con los investigadores y responsables de aplicaciones del proyecto donde se estipulan las funcionalidades deseables, dichas reuniones tuvieron lugar en el Departamente de Ingeniería informática de la Universidad de Santiago e Chile.

\section{Proceso de toma de requerimientos}
\label{sec:tomaDeRequerimientos}

Las primeras reuniones tuvieron un caracter exploratorio, en ellas se discutieron aspectos referentes al estado del arte y se acotaron los alcances del proyecto. En reuniones subsecuentes y mediante conversaciones con el equipo del proyecto FONDEF IDeA se encontraron los requisitos descritos en la tabla \ref{tab:requerimientos} mediante los cuales fue posible construir las historias de usuarios necesarias para la metodología que se está empleando. Los requisitos mostrados a continuación se corresponden con el identificador RFXX para los requerimientos funcionales, donde XX corresponde al número del requisitos y RNFYY, para el caso de los requerimientos no funcionales donde YY corresponde al número del requisito.

\begin{table}[H]
\centering
\caption[Requisitos encontrados]{Requisitos encontrados\\Fuente: Elaboración Propia, (2016)}
\label{tab:requerimientos}
\begin{tabular}{|c|l|c|}
\hline
\textbf{Identificador} & \multicolumn{1}{c|}{\textbf{Requisito}} & \textbf{Dependencia} \\ \hline
RNF00 & \begin{tabular}[c]{@{}l@{}}Se ha de implementar un sistema de procesamiento\\ de \textit\{streams\}.\end{tabular} & - \\ \hline
RNF01 & La fuente de datos debe ser \textit\{Twitter\}. & - \\ \hline
RNF02 & \begin{tabular}[c]{@{}l@{}}Debe obtenerse la información de \textit\{Twitter\}\\ en tiempo real.\end{tabular} & RNF01 \\ \hline
RNF03 & El sistema debe ser escalable & RNF00 \\ \hline
RNF04 & \begin{tabular}[c]{@{}l@{}}El sistema debe presentar una interfaz donde el\\ usuario pueda opearar.\end{tabular} & - \\ \hline
RNF05 & \begin{tabular}[c]{@{}l@{}}La interfaz debe poder recargar su información \\ sin actualizar la misma\end{tabular} & RNF04 \\ \hline
RNF06 & \begin{tabular}[c]{@{}l@{}}Los parámetros utilizados para operar el sistema \\ deben ser modificables.\end{tabular} & - \\ \hline
RNF07 & \begin{tabular}[c]{@{}l@{}}Se debe presentar un mapa para el despliegue de\\ la información.\end{tabular} & RNF04 \\ \hline
RNF08 & Cada categoría debe tener un icono particular. & RF05 \\ \hline
RNF09 & \begin{tabular}[c]{@{}l@{}}La información en el mapa debe agruparse según\\ nivel de acercamiento de éste.\end{tabular} & RNF08, RNF04 \\ \hline
RNF10 & \begin{tabular}[c]{@{}l@{}}Se debe poder elegir qué eventos se muestran por\\ categoría\end{tabular} & RNF08, RNF04 \\ \hline
RNF11 & \begin{tabular}[c]{@{}l@{}}Se deben mostrar estadísticas de procesamiento \\ para la consulta actual.\end{tabular} & RF02, RNF04 \\ \hline
RNF12 & \begin{tabular}[c]{@{}l@{}}Se debe mostrar desde donde viene un evento, es \\ decir, qué estado lo generó al visualizarlo en el\\ mapa.\end{tabular} & \begin{tabular}[c]{@{}c@{}}RNF10, RNF09,\\ RNF04\end{tabular} \\ \hline
RNF13 & \begin{tabular}[c]{@{}l@{}}Debe mostrarse una línea temporal para seleccionar\\ el intervalo.\end{tabular} & RF08 \\ \hline
RNF14 & \begin{tabular}[c]{@{}l@{}}Debe mostrarse un histograma con la cantidad de\\ eventos pasado por fecha.\end{tabular} & RF08 \\ \hline
RF00 & \begin{tabular}[c]{@{}l@{}}Se deben detectar las necesidades expresadas\\ en los \textit\{tweets\}\end{tabular} & \begin{tabular}[c]{@{}c@{}}RNF00, RNF01,\\ RNF02\end{tabular} \\ \hline
RF01 & \begin{tabular}[c]{@{}l@{}}Los operadores del sistema deben especificarse\\ como \textit\{bolts\}\end{tabular} & RNF00 \\ \hline
RF02 & Se debe poder especificar términos de búsqueda, & RNF00, RNF04, \\ \hline
RF03 & \begin{tabular}[c]{@{}l@{}}Los términos ingresados deben expandirse,\\ automáticamente, para abarcar nuesvos términos\\ útiles.\end{tabular} & RF03 \\ \hline
RF04 & Sólo se trabajaran \textit\{tweets\} en español. & RF01, RNF00 \\ \hline
RF05 & \begin{tabular}[c]{@{}l@{}}La detección de necesidades debe realizarse de \\ forma dinámica, no utilizando bolsas de palabras.\end{tabular} & - \\ \hline
RF06 & Se debe poder acceder a eventos pasados. & - \\ \hline
RF07 & Debe existir un sistema de persistencia de datos. & RF06 \\ \hline
RF08 & \begin{tabular}[c]{@{}l@{}}Se debe poder especificar un intervalo dentro del \\ cual mostrar datos.\end{tabular} & RF06 \\ \hline
RF09 & El sistema debe funcionar de forma continua & - \\ \hline
\end{tabular}
\end{table}

\section{Historias de usuario y criterios de aceptación}
\label{sec:historias}

Los requerimientos antes mencionados se agruparon para, según lo descrito por la metodología \textit{extreme programming}, como historias de usuario. Estas historias de usuario tienen la siguiente nomenclatura para su identificación: Aquellos que guarden relación con la aplicación de detección se identifican como 'HU-cXX', donde XX corresponde al número del requisito; 'HU-vYY' para aquellas que correspondan a la aplicación interfaz, donde, al igual que en el caso anterior, YY corresponden al número del requisito.

La Tabla \ref{tab:ReqTotales} presenta las historias de usuario correspondientes a los requerimientos de todo el sistema en general, sin hacer referencia al cómo está dividido éste.

\begin{table}[H]
\centering
\caption[Historias de usuario.]{Historias de usuario.\\Fuente: Elaboración Propia, (2016)}
\label{tab:ReqTotales}
\begin{tabular}{cl}
\hline
\textbf{Identificador}       & \multicolumn{1}{c}{\textbf{Historia de usuario}}                                                                                                                                                                                                                                                                                                                                      \\ \hline
\multicolumn{1}{|c|}{HU-c00} & \multicolumn{1}{l|}{\begin{tabular}[c]{@{}l@{}}Como cliente quiero capturar necesidades de la población en tiempo real \\ cuando el país se encuentre en un escenario de catástrofe natural para poder \\ contar con información para asistir a la población afectada.\end{tabular}}                                                                                                  \\ \hline
\multicolumn{1}{|c|}{HU-c01} & \multicolumn{1}{l|}{\begin{tabular}[c]{@{}l@{}}Como cliente quiero que las necesidades detectadas se recojan desde la \\ información generadas en redes sociales redes sociales para que las personas\\ sean la fuente primaria.\end{tabular}}                                                                                                                                        \\ \hline
\multicolumn{1}{|c|}{HU-c02} & \multicolumn{1}{l|}{\begin{tabular}[c]{@{}l@{}}Como cliente quiero que la búsqueda de necesidades se vea enriquecida para\\  abarcar nuevos términos de búsqueda para abarcar un conjunto mayor de \\ información.\end{tabular}}                                                                                                                                                      \\ \hline
\multicolumn{1}{|c|}{HU-v00} & \multicolumn{1}{l|}{\begin{tabular}[c]{@{}l@{}}Como usuario quiero una interfaz donde pueda visualizar el comportamiento \\ del sistema de detección para poder interactuar con el.\end{tabular}}                                                                                                                                                                                     \\ \hline
\multicolumn{1}{|c|}{HU-v01} & \multicolumn{1}{l|}{\begin{tabular}[c]{@{}l@{}}Como cliente quiero que las necesidades detectadas puedan ser asociadas a \\ un punto en un mapa geográfico para poder identificar el lugar físico de su \\ fuente.\end{tabular}}                                                                                                                                                      \\ \hline
\multicolumn{1}{|c|}{HU-v02} & \multicolumn{1}{l|}{\begin{tabular}[c]{@{}l@{}}Como usuario quiero que puedan aplicarse filtros a la visualización de los \\ puntos de modo que según la distancia entre ellos, cuáles se quieran mostrar\\ y el nivel de acercamiento que tenga el mapa se entreguen diferentes formas\\ de mostrar la información para que la información se visualice con facilidad.\end{tabular}} \\ \hline
\multicolumn{1}{|c|}{HU-v03} & \multicolumn{1}{l|}{\begin{tabular}[c]{@{}l@{}}Como usuario quiero que la visualización de eventos se realice en tiempo \\ real para tomar decisiones rápidas cuando la situación lo amerite.\end{tabular}}                                                                                                                                                                           \\ \hline
\multicolumn{1}{|c|}{HU-v04} & \multicolumn{1}{l|}{\begin{tabular}[c]{@{}l@{}}Como usuario quiero visualizar eventos pasados, además quiero poder \\ seleccionar un intervalo de tiempo y que el sistema muestre todos los \\ eventos que se hayan detectado dentro de aquel intervalo de modo que \\ pueda realizarse una análisis a posteriori de la emergencia.\end{tabular}}                                     \\ \hline
\multicolumn{1}{|c|}{HU-v05} & \multicolumn{1}{l|}{\begin{tabular}[c]{@{}l@{}}Como usuario quiero poder especificar términos de búsqueda para acotar la\\  búsqueda a aquel contenido que contenga elementos que se correspondan \\ con ellos.\end{tabular}}                                                                                                                                                         \\ \hline
\multicolumn{1}{|c|}{HU-v06} & \multicolumn{1}{l|}{\begin{tabular}[c]{@{}l@{}}Como usuario quiero que cada punto, correspondiente a una necesidad \\ específica, tenga un diseño particular fácilmente identificable.\end{tabular}}                                                                                                                                                                                  \\ \hline
\multicolumn{1}{|c|}{HU-07}  & \multicolumn{1}{l|}{\begin{tabular}[c]{@{}l@{}}Como usuario quiero que sea posible visualizar estadísticas del procesamiento\\ de la aplicación por consulta.\end{tabular}}                                                                                                                                                                                                           \\ \hline
\multicolumn{1}{|c|}{HU-08}  & \multicolumn{1}{l|}{\begin{tabular}[c]{@{}l@{}}Como usuario quiero poder modificar cuánto tiempo se visualizará un evento \\ antes de que sea considerado antiguo y cada cuánto tiempo se ha de añadir la\\ información de los nuevos eventos.\end{tabular}}                                                                                                                               \\ \hline
\end{tabular}
\end{table}

Estas historias de usuario se corresponden con los criterios de aceptación descritos en la Tabla \ref{tab:criteriosAceptacion} que se presenta a continuación.

\begin{table}[H]
\centering
\caption[Criterios de aceptación.]{Criterios de aceptación.\\Fuente: Elaboración Propia, (2016)}
\label{tab:criteriosAceptacion}
\begin{tabular}{|c|l|}
\hline
\textbf{Identificador} & \multicolumn{1}{c|}{\textbf{Criterio de aceptación}} \\ \hline
HU-c00 & \begin{tabular}[c]{@{}l@{}}· Debe contruirse el sistema de procesamiento de stream.\\ · Debe capturarse la información desde Twitter.\\ · Debe usarse la información que llega al sistema y no almacenarla para\\ trabajarla por lotes.\\ · El sistema debe concluir con un dato etiquetado en la base de datos.\end{tabular} \\ \hline
HU-c01 & · Twitter debe ser desde donde se obtienen los datos de entrada del sistema. \\ \hline
HU-c02 & · Deben implementarse técnicas para realizar expansión de la consulta. \\ \hline
HU-v00 & \begin{tabular}[c]{@{}l@{}}· Debe contarse con una interfaz que muestre los eventos detectados por \\ el sistema.\end{tabular} \\ \hline
HU-v01 & \begin{tabular}[c]{@{}l@{}}· El mapa de eventos debe mostrar los eventos, asociados a un punto \\ geográfico.\end{tabular} \\ \hline
HU-v02 & \begin{tabular}[c]{@{}l@{}}· Debe poder filtrarse por modos de agrupamiento.\\ · Debe poder filtrarse por categorías.\\ · El mapa debe permitir modificar su nivel de acercamiento.\end{tabular} \\ \hline
HU-v03 & · El mapa debe actualizarse automáticamente. \\ \hline
HU-v04 & \begin{tabular}[c]{@{}l@{}}· Debe poder seleccionarse el intervalo.\\ · Los eventos en el mapa deben modificarse según el intervalo que se \\ seleccione.\\ · Debe seleccionarse el intervalo mediante una línea de tiempo.\\ · Debe mostrarse un histograma con los eventos pasados presentes en el\\ sistema.\end{tabular} \\ \hline
HU-v05 & \begin{tabular}[c]{@{}l@{}}· Debe existir un lugar donde especificar éstos términos.\\ · El sistema debe mostrar qué términos se están utilizando para la \\ búsqueda actual.\end{tabular} \\ \hline
HU-v06 & · Cada evento de categorización diferente debe tener un icono particular. \\ \hline
HU-v07 & \begin{tabular}[c]{@{}l@{}}· Deben mostrarse la cantidad de usuarios diferentes que han emitido\\ estados detectados.\\ · Deben mostrarse la cantidad de necesidades detectadas.\\ · Deben mostrarse la cantidad de eventos procesados.\end{tabular} \\ \hline
HU-v08 & \begin{tabular}[c]{@{}l@{}}· Debe existir un lugar donde realizar el cambio de parámetros.\\ · Debe ser explícito el cambio de los parámetros de funcionamiento en\\ la interfaz.\end{tabular} \\ \hline
\end{tabular}
\end{table}