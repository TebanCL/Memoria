\chapter{Requerimientos}
\label{cap:Requerimientos}

Éste capítulo detallará los requisitos de las aplicaciones, descritos como historias de usuario. Éstos señalan las necesidades de los clientes.

El cliente es uno de los clientes del proyecto FONDEF IDeA, código ID15I10560, proyecto orientado a generar herramientas para la gestión de desastres naturales. Los requerimientos se levantan en reuniones con los investigadores y responsables de aplicaciones del proyecto donde se estipulan las funcionalidades deseables, dichas reuniones tuvieron lugar en el Departamente de Ingeniería informática de la Universidad de Santiago e Chile.

\section{Proceso de toma de requerimientos}
\label{sec:tomaDeRequerimientos}




\section{Historias de usuario y criterios de aceptación}
\label{sec:historias}

Los requerimientos antes mencionados se agruparon para, según lo descrito por la metodología \textit{extreme programing}, como historias de usuario. Estas historias de usuario tienen la siguiente nomenclatura para su identificación: Aquellos que guarden relación con la aplicación de detección se identifican como 'HU-cXX' donde XX corresponde al número del requisito; 'HU-vYY' para aquellas que correspondan a la aplicación interfáz donde, al igual que en el caso anterior, YY corresponden al número del requisito.

La Tabla \ref{tab:ReqTotales} presenta las historias de usuario correspondientes a los requerimientos de todo el sistema en general, sin hacer referencia al cómo está dividido éste.

\begin{table}[H]
\centering
\caption[Historias de usuario.]{Historias de usuario.\\Fuente :Elaboración Propia, (2016)}
\label{tab:ReqTotales}
\begin{tabular}{cl}
\hline
\textbf{Identificador}       & \multicolumn{1}{c}{\textbf{Historia de usuario}}                                                                                                                                                                                                                                                                                                                                      \\ \hline
\multicolumn{1}{|c|}{HU-c00} & \multicolumn{1}{l|}{\begin{tabular}[c]{@{}l@{}}Como cliente quiero capturar necesidades de la población en tiempo real \\ cuando el país se encuentre en un escenario de catástrofe natural para poder \\ contar con información para asistir a la población afectada.\end{tabular}}                                                                                                  \\ \hline
\multicolumn{1}{|c|}{HU-c01} & \multicolumn{1}{l|}{\begin{tabular}[c]{@{}l@{}}Como cliente quiero que las necesidades detectadas se recojan desde la \\ información generadas en redes sociales redes sociales para que las personas\\ sean la fuente primaria.\end{tabular}}                                                                                                                                        \\ \hline
\multicolumn{1}{|c|}{HU-c02} & \multicolumn{1}{l|}{\begin{tabular}[c]{@{}l@{}}Como cliente quiero que la búsqueda de necesidades se vea enriquecida para\\  abarcar nuevos términos de búsqueda para abarcar un conjunto mayor de \\ información.\end{tabular}}                                                                                                                                                      \\ \hline
\multicolumn{1}{|c|}{HU-v00} & \multicolumn{1}{l|}{\begin{tabular}[c]{@{}l@{}}Como usuario quiero una interfaz donde pueda visualizar el comportamiento \\ del sistema de detección para poder interactuar con el.\end{tabular}}                                                                                                                                                                                     \\ \hline
\multicolumn{1}{|c|}{HU-v01} & \multicolumn{1}{l|}{\begin{tabular}[c]{@{}l@{}}Como cliente quiero que las necesidades detectadas puedan ser asociadas a \\ un punto en un mapa geográfico para poder identificar el lugar físico de su \\ fuente.\end{tabular}}                                                                                                                                                      \\ \hline
\multicolumn{1}{|c|}{HU-v02} & \multicolumn{1}{l|}{\begin{tabular}[c]{@{}l@{}}Como usuario quiero que puedan aplicarse filtros a la visualización de los \\ puntos de modo que según la distancia entre ellos, cuáles se quieran mostrar\\ y el nivel de acercamiento que tenga el mapa se entreguen diferentes formas\\ de mostrar la información para que la información se visualice con facilidad.\end{tabular}} \\ \hline
\multicolumn{1}{|c|}{HU-v03} & \multicolumn{1}{l|}{\begin{tabular}[c]{@{}l@{}}Como usuario quiero que la visualización de eventos se realice en tiempo \\ real para tomar decisiones rápidas cuando la situación lo amerite.\end{tabular}}                                                                                                                                                                           \\ \hline
\multicolumn{1}{|c|}{HU-v04} & \multicolumn{1}{l|}{\begin{tabular}[c]{@{}l@{}}Como usuario quiero visualizar eventos pasados, además quiero poder \\ seleccionar un intervalo de tiempo y que el sistema muestre todos los \\ eventos que se hayan detectado dentro de aquel intervalo de modo que \\ pueda realizarse una análisis a posteriori de la emergencia.\end{tabular}}                                     \\ \hline
\multicolumn{1}{|c|}{HU-v05} & \multicolumn{1}{l|}{\begin{tabular}[c]{@{}l@{}}Como usuario quiero poder especificar términos de búsqueda para acotar la\\  búsqueda a aquel contenido que contenga elementos que se correspondan \\ con ellos.\end{tabular}}                                                                                                                                                         \\ \hline
\multicolumn{1}{|c|}{HU-v06} & \multicolumn{1}{l|}{\begin{tabular}[c]{@{}l@{}}Como usuario quiero que cada punto, correspondiente a una necesidad \\ específica, tenga un diseño particular fácilmente identificable.\end{tabular}}                                                                                                                                                                                  \\ \hline
\multicolumn{1}{|c|}{HU-07}  & \multicolumn{1}{l|}{\begin{tabular}[c]{@{}l@{}}Como usuario quiero que sea posible visualizar estadísticas del procesamiento\\ de la aplicación por consulta.\end{tabular}}                                                                                                                                                                                                           \\ \hline
\multicolumn{1}{|c|}{HU-08}  & \multicolumn{1}{l|}{\begin{tabular}[c]{@{}l@{}}Como usuario quiero poder modificar cuánto tiempo se visualizará un evento \\ antes de que sea considerado antiguo y cada cuánto tiempo se añadirá la\\ información de los nuevos eventos.\end{tabular}}                                                                                                                               \\ \hline
\end{tabular}
\end{table}

Estas historias de usuario se corresponden con los criterios de aceptación descritos en la Tabla \ref{tab:criteriosAceptacion} que se presenta a continuación.

\textbf{Faltan criterios, conversarlo con profes}

