\chapter{Requerimientos}
\label{cap:Requerimientos}

Éste capítulo detalla los requisitos de las aplicaciones, descritos como historias de usuario. Éstos señalan las necesidades de los clientes.

El cliente corresponde a miembros del equipo que lidera el proyecto FONDEF IDeA, código ID15I10560, proyecto orientado a generar herramientas para la gestión de desastres naturales. Los requerimientos se levantan en reuniones con los investigadores y responsables de aplicaciones del proyecto donde se estipulan las funcionalidades deseables, dichas reuniones tuvieron lugar en el Departamente de Ingeniería informática de la Universidad de Santiago e Chile.

\section{Proceso de toma de requerimientos}
\label{sec:tomaDeRequerimientos}

Las primeras reuniones tuvieron un caracter exploratorio, en ellas se discutieron aspectos referentes al estado del arte y se acotaron los alcances del proyecto. En reuniones subsecuentes y mediante conversaciones con el equipo del proyecto FONDEF IDeA se encontraron los requisitos descritos en la tabla \ref{tab:requerimientos} mediante los cuales fue posible construir las historias de usuarios necesarias para la metodología que se está empleando. Los requisitos mostrados a continuación se corresponden con el identificador RFXX para los requerimientos funcionales, donde XX corresponde al número del requisitos y RNFYY, para el caso de los requerimientos no funcionales donde YY corresponde al número del requisito.

\section{Historias de usuario y criterios de aceptación}
\label{sec:historias}

Los requerimientos antes mencionados se agruparon para, según lo descrito por la metodología \textit{extreme programming}, como historias de usuario. Estas historias de usuario tienen la siguiente nomenclatura para su identificación: Aquellos que guarden relación con la aplicación de detección se identifican como 'HU-cXX', donde XX corresponde al número del requisito; 'HU-vYY' para aquellas que correspondan a la aplicación interfaz, donde, al igual que en el caso anterior, YY corresponden al número del requisito.

La Tabla \ref{tab:ReqTotales} presenta las historias de usuario correspondientes a los requerimientos de todo el sistema en general, sin hacer referencia al cómo está dividido éste.

\begin{table}[H]
\centering
\caption[Resumen de las historias de usuario.]{Resumen de las historias de usuario.\\Fuente: Elaboración Propia, (2016)}
\label{tab:ReqTotales}
\resizebox{\textwidth}{!}{%
\begin{tabular}{|c|l|}
\hline
\textbf{Identificador} & \multicolumn{1}{c|}{\textbf{Historia de usuario}} \\ \hline
HU-c00 & \begin{tabular}[c]{@{}l@{}}Como cliente quiero capturar necesidades de la población en tiempo real \\ cuando el país se encuentre en un escenario de catástrofe natural para\\ poder contar con información para asistir a la población afectada.\end{tabular} \\ \hline
HU-c01 & \begin{tabular}[c]{@{}l@{}}Como cliente quiero que las necesidades detectadas se recojan desde la \\ información generadas en redes sociales como Twitter, dado su caracter \\ informativo.\end{tabular} \\ \hline
HU-c02 & \begin{tabular}[c]{@{}l@{}}Como cliente quiero que la búsqueda de necesidades pueda ser extendida \\ automáticamente para encontrar información adicional que pueda ser de \\ apoyo.\end{tabular} \\ \hline
HU-c03 & \begin{tabular}[c]{@{}l@{}}Como usuario deseo poder incluir diferentes modelos de clasificación \\ dependiendo de las características del evento a analizar.\end{tabular} \\ \hline
HU-c04 & \begin{tabular}[c]{@{}l@{}}Como cliente quiero almacenar datos históricos para poder realizar \\ análisis mas a fondo, incluso entre diferentes eventos.\end{tabular} \\ \hline
HU-v00 & \begin{tabular}[c]{@{}l@{}}Como usuario quiero una interfaz basada en un mapa geográfico donde\\ se pueda interactuar con la información generada.\end{tabular} \\ \hline
HU-v01 & \begin{tabular}[c]{@{}l@{}}Como cliente quiero que las necesidades detectadas puedan ser asociadas \\ aun punto en un mapa geográfico para poder identificar el lugar físico de \\ su fuente, esta ubicación debe ser de manera automática, incluso si no se \\ cuenta con los datos de geoubicación.\end{tabular} \\ \hline
HU-v02 & \begin{tabular}[c]{@{}l@{}}Como usuario quiero que puedan aplicarse filtros a la visualización de \\ los puntos de modo que según la distancia entre ellos, cuáles se quieran \\ mostrar y el nivel de acercamiento que tenga el mapa se entreguen \\ diferentes formas de mostrar la información para que la información se \\ visualice con facilidad.\end{tabular} \\ \hline
HU-v03 & \begin{tabular}[c]{@{}l@{}}Como usuario quiero que la visualización de eventos se realice en tiempo\\ real para tomar decisiones rápidas cuando la situación lo amerite.\end{tabular} \\ \hline
HU-v04 & \begin{tabular}[c]{@{}l@{}}Como usuario quiero visualizar eventos pasados, además quiero poder \\ seleccionar un intervalo de tiempo y que el sistema muestre todos los \\ eventos que se hayan detectado dentro de aquel intervalo de modo que \\ pueda realizarse una análisis a posteriori de la emergencia.\end{tabular} \\ \hline
HU-v05 & \begin{tabular}[c]{@{}l@{}}Como usuario quiero poder especificar términos de búsqueda para \\ acotar la búsqueda sólo a aquellos datos que contengan elementos\\ relevantes para la situación a analizar.\end{tabular} \\ \hline
HU-v06 & \begin{tabular}[c]{@{}l@{}}Como usuario quiero que cada punto, correspondiente a una necesidad \\ específica, tenga un diseño particular fácilmente identificable.\end{tabular} \\ \hline
HU-v07 & \begin{tabular}[c]{@{}l@{}}Como usuario quiero que sea posible visualizar estadísticas del \\ procesamiento de la aplicación por consulta.\end{tabular} \\ \hline
HU-v08 & \begin{tabular}[c]{@{}l@{}}Como usuario quiero poder modificar cuánto tiempo se visualizará un \\ evento antes de que sea considerado antiguo y cada cuánto tiempo se ha \\ de añadir la información de los nuevos eventos.\end{tabular} \\ \hline
\end{tabular}%
}
\end{table}

Estas historias de usuario se corresponden con los criterios de aceptación descritos en la Tabla \ref{tab:criteriosAceptacion} que se presenta a continuación.

\begin{table}[H]
\centering
\caption[Criterios de aceptación.]{Criterios de aceptación.\\Fuente: Elaboración Propia, (2016)}
\label{tab:criteriosAceptacion}
\begin{tabular}{|l|l|}
\hline
\multicolumn{1}{|c|}{\textbf{Identificador}} & \multicolumn{1}{c|}{\textbf{Criterio de aceptación}} \\ \hline
HU-c00 & \begin{tabular}[c]{@{}l@{}}· Debe construirse una topologia para Storm capaz de detectar \\ las necesidades.\\ · Debe crearse un operador capaz de extraer información desde\\ una fuente de datos basada en texto.\\ · Debe almacenar informacion historica para una ventana de \\ tiempo definida por el usuario.\\ · Los elementos almacenados deben estar correctamente \\ etiquetados.\end{tabular} \\ \hline
HU-c01 & \begin{tabular}[c]{@{}l@{}}· Twitter debe ser desde donde se obtienen los datos de entrada \\ del sistema.\end{tabular} \\ \hline
HU-c02 & \begin{tabular}[c]{@{}l@{}}· Deben implementarse técnicas para realizar expansión de la \\ consulta.\end{tabular} \\ \hline
HU-c03 & · El modelo de clasificación debe poder modificarse. \\ \hline
HU-c04 & · Los datos deben estar almacenados en un repositorio local. \\ \hline
HU-v00 & \begin{tabular}[c]{@{}l@{}}· Debe contarse con una interfaz que muestre los eventos \\ detectados por el sistema.\end{tabular} \\ \hline
HU-v01 & \begin{tabular}[c]{@{}l@{}}· El mapa de eventos debe mostrar los eventos, asociados a un \\ punto geográfico.\end{tabular} \\ \hline
HU-v02 & \begin{tabular}[c]{@{}l@{}}· Se debe poder filtrar por modos de agrupamiento.\\ · Se debe poder filtrar por categorías.\\ · El mapa debe permitir modificar su nivel de acercamiento.\end{tabular} \\ \hline
HU-v03 & · El mapa debe actualizarse automáticamente. \\ \hline
HU-v04 & \begin{tabular}[c]{@{}l@{}}· Debe poder seleccionarse el intervalo.\\ · Los eventos en el mapa deben modificarse según el intervalo \\ que se seleccione.\\ · Debe seleccionarse el intervalo mediante una línea de tiempo.\\ · Debe mostrarse un histograma con los eventos pasados \\ presentes en el sistema.\end{tabular} \\ \hline
HU-v05 & \begin{tabular}[c]{@{}l@{}}· Debe existir un lugar donde especificar éstos términos.\\ · El sistema debe mostrar qué términos se están utilizando para\\ la búsqueda actual.\end{tabular} \\ \hline
HU-v06 & \begin{tabular}[c]{@{}l@{}}· Cada evento de categorización diferente debe tener un icono \\ particular.\end{tabular} \\ \hline
HU-v07 & \begin{tabular}[c]{@{}l@{}}· Deben mostrarse la cantidad de usuarios diferentes que han \\ emitido estados detectados.\\ · Deben mostrarse la cantidad de necesidades detectadas.\\ · Deben mostrarse la cantidad de eventos procesados.\end{tabular} \\ \hline
HU-v08 & \begin{tabular}[c]{@{}l@{}}· Debe existir un lugar donde realizar el cambio de parámetros.\\ · Debe ser explícito el cambio de los parámetros de \\ funcionamiento en la interfaz.\end{tabular} \\ \hline
\end{tabular}
\end{table}